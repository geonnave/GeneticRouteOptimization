\section{Introdução}


Os Problemas de Roteamento tem sido amplamente estudados nas últimas décadas, visto que se mostram pertinentes e podem ser aplicáveis a diversos meios como: transportes, logística e redes de computadores. Como esse tipo de problema normalmente não costuma apresentar solução computacionalmente viável \cite{paulo2011agprv}, técnicas inteligentes são indicadas para otimizar o desempenho de algoritmos cujo objetivo é resolvê-los. Assim, tendo em vista o fato de que o foco deste trabalho é o desenvolvimento de um sistema computacional de otimização utilizando um algoritmo genético, este artigo pretende demonstrar e explicar o funcionamento tal sistema, bem como seus resutados, aplicado ao problema de roteamento.
	
Um Algoritmo Genético é uma técnica computacional de otimização cujo funcionamento remete ao princípio Darwiniano de Seleção Natural \cite{marco1999agpa}. Segundo esse princípio, os indivíduos de uma determinada espécie tendem a evoluir a cada nova geração. Isso se deve ao fato de que os elementos mais aptos a sobreviver ao meio em que vivem são os que tendem a conseguir procriar, ou seja, os novos indivíduos sempre tendem a possuir os códigos genéticos dos indivíduos mais aptos da geração anterior. Os algoritmos genéticos tentam imitar esse comportamento em problemas de otimização, de forma que: cada solução possível é encarada como um indivíduo (ou cromossomo) e cada atributo da solução é chamada gene.

